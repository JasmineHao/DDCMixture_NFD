\documentclass{article}

\usepackage{amssymb,amsmath,amsfonts,eurosym,geometry,ulem,graphicx,caption,color,setspace,sectsty,comment,footmisc,caption,natbib,pdflscape,subfigure,array,hyperref,upgreek,bbm}

\normalem


\usepackage{amsthm}
\usepackage{amssymb}
\usepackage{graphicx}
\usepackage{amsmath}
\usepackage{verbatim}
\usepackage{setspace}
\usepackage{ulem}
\usepackage{textpos}
\usepackage{changepage}
\usepackage{url}
%\usepackage{multirow}


%\usepackage{hyperref}
%\usepackage{xcolor}
%\hypersetup{colorlinks, linkcolor=blue, citecolor=blue}

%\renewcommand{\sout}[1]{}
%\renewcommand{\textbf}{}
%\renewcommand{\bf}{}

\newcommand{\tsout}[1]{\text{\sout{$#1$}}}


\tolerance=5000
\newtheorem{remark}{Remark}
\def\mb{\mathbf}
\def\iid{\mathrm{i.i.d. }}
\def\bs{\boldsymbol}
\def\tbf{\textbf}
\def\t{^{\top}}
\def\bSig{\bs{\Sigma}}
\newcommand{\mcitet}[1]{\mbox{\citet{#1}}}
\newcommand{\mcitep}[1]{\mbox{\citep{#1}}}

\DeclareMathOperator{\vect}{vec}
\DeclareMathOperator{\vecth}{vech}


\onehalfspacing
\newtheorem{theorem}{Theorem}
\newtheorem{corollary}[theorem]{Corollary}
\newtheorem{proposition}{Proposition}
\newtheorem{lemma}[theorem]{Lemma}
\newtheorem{example}{Example}
\newtheorem{definition}{Definition}
%\newenvironment{proof}[1][Proof]{\noindent\textbf{#1.} }{\ \rule{0.5em}{0.5em}}

\newcommand{\ind}{\mathbbm{1}}

\newcommand{\R}{\mathbb{R}}
\newtheorem{assumption}{Assumption}

\newtheorem{hyp}{Hypothesis}
\newtheorem{subhyp}{Hypothesis}[hyp]
\renewcommand{\thesubhyp}{\thehyp\alph{subhyp}}

\newcommand{\red}[1]{{\color{red} #1}}
\newcommand{\blue}[1]{{\color{blue} #1}}

\newcolumntype{L}[1]{>{\raggedright\let\newline\\arraybackslash\hspace{0pt}}m{#1}}
\newcolumntype{C}[1]{>{\centering\let\newline\\arraybackslash\hspace{0pt}}m{#1}}
\newcolumntype{R}[1]{>{\raggedleft\let\newline\\arraybackslash\hspace{0pt}}m{#1}}


\title{The estimators}
\author{Hiro Kasahara}
\date{\today}

\begin{document}

\maketitle
\section{Estimators}
The estimators assume that the conditional choice probabilities are correctly estimated in the first step.
Therefore, we obtain the consistent estimator of $F$ and $P$: $\hat{F}$ and $\hat{P}$ from the data.
We define the mappings as $\Phi : \mathcal{B}_{P} \to \mathcal{B}_{V}$ and $\Lambda: \mathcal{B}_{V} \to \mathcal{B}_{P}$.
The estimators construct different $\Phi$ mappings.
\begin{itemize}
  \item Hotz-Miller estimator \begin{equation}
  \begin{split}
          \Phi_{HM}(P;\theta) & = (I - \beta F^e(P))^{-1}( u^e(P;\theta) + e^e(P) ), \\
          \text{ where } F^e(P) & = \left[ \sum_{d \in \mathcal{D} } p(d,z) f(z'|z,d)\right]_{z,z' \in \mathcal{Z}}
          \text{ and } u^e(P;\theta)  = \left[ \sum_{d \in \mathcal{D} } p(d,z) u(d,z;\theta) \right]_{z \in \mathcal{Z}} \\
          \Psi_{HM}(P;\theta) & = \Lambda(\Phi_{HM}(P;\theta))
  \end{split}
  \end{equation}
  \item Euler equation estimator \begin{equation}
  \begin{split}
          \Phi_{EE}(P;\theta) & = (I - \beta F(0))^{-1}( u(0;\theta) + e(0,P))\\
          \Psi_{EE}(P;\theta) & = \Lambda(\Phi_{EE}(P;\theta))
  \end{split}
  \end{equation}
  \item Finite Dependence \[ \begin{split}
    \Psi_{FD}(d,x;\theta,P) = \frac{u(d,x;\theta) + \beta \sum_{x'|x}f(x'|x,d)(u(0,x';\theta) + e(0,x,P) ) }{ \sum_{d'} u(d',x;\theta) + \beta \sum_{x'|x}f(x'|x,d')(u(0,x';\theta) + e(0,x,P) )}
  \end{split} \]
\end{itemize}
\section{Two-step estimators}

\begin{itemize}
  \item In the first step, obtain non-parametric estimator of the conditional choice probabilities $\hat{p}(d,x)$.
  \item In the second step, use the maximum likelihood estimator and obtain $\hat{\theta} = \arg \max_{\theta} \sum_{i=1}^N \sum_{t=1}^T \Psi(d_{it},x_{it},\theta,\hat{P})$.
\end{itemize}
\end{document}
